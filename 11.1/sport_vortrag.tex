\documentclass{article}
\usepackage[utf8]{inputenc}

\title{Handout: Spielbedingungen}
\author{Felix Rady, Marius Cramer}
\date{\today}

\begin{document}
\maketitle

\section{Spielfläche}
\begin{itemize}
  \item 18m x 9m
  \item 9m x 9m pro Spielhälfte
  \item Freizone um das Spielfeld herum
\end{itemize}

\section{Netz}
\begin{itemize}
  \item Über der Mittellinie
  \item Bei Männern obere Kante 2.43m hoch
  \item Bei Frauen obere Kante 2.24m hoch
  \item Netz ist 1m hoch, 9.5m bis 10m breit
  \item Netzpfosten 0.5m - 1m ausserhalb des Spielfelds
\end{itemize}

\section{Linien}
\begin{itemize}
  \item Mittellinie
  \item Vorderzone: jeweils 3m hinter Mittellinie
  \item Restlicher Teil Hinterzone
  \item Linien sind 5cm breit
\end{itemize}

\section{Zonen und Flächen}
\begin{itemize}
  \item Spielfeld von 3m vreiter Freizone umgeben
  \item In Halle wird 8m x 8m innenraum empfohlen
  \item Hinter der Grundlinie ist Aufschlagszone
  \item Bei kleinen Vereinen und niedrigen Ligen sind Abmessungen kleiner
\end{itemize}

\section{Ball}
\begin{itemize}
  \item Nahtlos, leicht aufgepolstertes Leder oder Kusntstoff
  \item Luftgefüllte Gummiblase im Inneren
  \item Umfang: 65cm bis zu 67cm
  \item Gewicht: 260g bis zu 280g
\end{itemize}

\section{Mannschaft}
\begin{itemize}
  \item Normale Aufstellung: 6 Spieler pro Team
  \item Bekommt eine Mannschaft Aufschlagsrecht rotieren die Spieler eine Position im Uhrzeigersinn
  \item meist:
  \begin{itemize}
    \item Zwei Mittelblocker (Position 3, 6)
    \item Zwei Aussenangreifer (Position 4, 5)
    \item Zwei Zuspieler (Position 1, 2)
  \end{itemize}
\end{itemize}

\section{Spielverlauf}
\begin{enumerate}
  \item Aufschlag durch Team 1
  \item Team 2 nimmt an und kontrolliert den Ball, greifft dann wieder an
  \item Nun kann Team 1 entweder den Ball am Netz blocken oder den nächsten Angriff aufbauen
\end{enumerate}

\section{Auswechslungen}
\begin{itemize}
  \item Pro Satz und Mannschaft 6 Auswechslungen
  \item "Sobald ein Spieler für einen anderen eingewechselt wurde, kann er auch nur für diesen wieder ausgewechselt werden (so genannter Rückwechsel), mit Ausnahme des Liberos. Danach ist in diesem Satz für diese beiden Spieler das Wechselkontingent erschöpft, mit der Folge, dass der Startspieler den Satz zu Ende spielen und der Ersatzspieler bis zum nächsten Satz auf der Bank Platz nehmen muss."
\end{itemize}

\section*{Quellen}
\begin{itemize}
  \item www.wikipedia.org
\end{itemize}

\end{document}
