\documentclass{article}

\usepackage[utf8]{inputenc}
\usepackage{xcolor}
\usepackage{textcomp}
\usepackage[T1]{fontenc}

\title{ethik}
\author{marius cramer}
\date{21-08-2018}

\definecolor{myGreen}{RGB}{97, 154, 76}
\definecolor{myRed}{RGB}{76, 72, 72}
\definecolor{myBlue}{RGB}{77, 130, 185}
\definecolor{myOrange}{RGB}{221, 137, 51}

\begin{document}
\maketitle

\tableofcontents

\newpage

\section{Vortrag}
\begin{enumerate}
  \item Thema: Religionskritik
  \item 2-3 Schüler
  \item 15-20 Minuten
  \item Handouts
\end{enumerate}

\section{moralische Handlungsbegründungen}
\begin{enumerate}
  \item Deontologie
  \item Teleologie
\end{enumerate}
\textbf{Deontologie:}
\begin{itemize}
  \item eine Handlung wird nur nach der zugrunde liegenden Absicht bewertet
  \item Die zugrunde liegende Absicht bestimmt den Wert der Handlung, Folgen spielen keine Rolle.
\end{itemize}

\bigskip

\begin{itemize}
  \item
  \item
\end{itemize}

\bigskip

Emmanuel Kant, Imperativ, handle stets nach der Maxine, von der du wollen kannst, dass auch alle anderen nach ihr Handeln.




\end{document}
