\documentclass[12pt]{article}

\title{Excercise 5: Life's Common Core in 20 years}
\author{Marius Cramer}
\date{\today}

\begin{document}
\maketitle

Due to technological and mechanical advancements in science and consumer technolgies the common life advice will change dramatically in the next two decades even more so than it already changed greatly in the last four decades due to the rise of computers. Teachers and parents will have to adapt to it and incorporate computers into lessons and the whole education system, otherwise they won't be able to keep up. Another thing to change in the next two decades is currency. I think in 15 years currency will be completely digitally managed and handled. This means parents will have to find new strategies to teach money management and frugality to their kids as it will not be the same as with physical money just because of psychological effects. Also parents should be able to code properly by then as it will be a major part of common life in 20 years in my view. Schools will have to push more into technologies, especially in the field of computer sciences and teach programming more efficiently and more contemporary. Coding or programming will be a necessary skill to have as it serves as a way to maintain freedom of speech and personal autonomy for the individual person. Both of these things will rise in significance as the outreach of technology and especially the internet in todays form grows into political and world economical areas like elections or wallstreet. Citizens of big goverments like the United States of America will need the skills of protecting themselves from cyber attacks from within and outside of their own government and in two decades this can only be achieved by knowing how to code. Parents as well as teachers will stick to the same main principles as parents and teachers today or 50 years ago but they will need to incorporate technologies more or they \textit{will} fall behind the times.

\end{document}
