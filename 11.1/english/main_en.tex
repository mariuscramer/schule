\documentclass{article}

\usepackage[utf8]{inputenc}
\usepackage{xcolor}
\usepackage{textcomp}
\usepackage[T1]{fontenc}

\title{englisch}
\author{marius cramer}
\date{16-08-2018}

\definecolor{myGreen}{RGB}{97, 154, 76}
\definecolor{myRed}{RGB}{176, 72, 72}
\definecolor{myBlue}{RGB}{77, 130, 185}
\definecolor{myOrange}{RGB}{221, 137, 51}

\begin{document}
\maketitle

\tableofcontents

\newpage

\section{\textbf{Einführung}}
\subsection{A curriculum for life}
\begin{enumerate}
  \item Discuss the activities.
  \begin{itemize}
    \item If you race to the top you might just miss out on opportunities to diversify your life. Missing these chances may mean you will be short on life experience later in life when it comes to being better than your rivals once at the top. van Ogtrop mentions you shouldn't knock enemies out of our way because it could result in you making unnecessary enemies on your way to the top. This tip seems critical for all ages, it applies to most things in life. It will also teach you to be more wholesome and kind to your surrounding people and yourself.
    \item Keeping an unenjoyable job means accumulating very important life experience on getting through the rough parts of life. Your parents most likely already know this and want you to learn the same lessons. Always keep in mind your parents only want the best for you, in most cases anyway. This is also a tip which will always remain up to date and important.
    \item Going without a phone for a weekend or even a multiple days will teach you to not rely solely on your electronic media. This is not as relevant to most teenagers as adults think because not every teen is addicted to their phone and can easily go without for some time. Aside from all this learning to stay away from your phone once a week or so for a few hours and doing something else entertainment-wise is a good lesson to be learned and will enrich your life in the long-term.
  \end{itemize}
  \item We believe the second and third tip are not very relevant to our lives as we do not have a job nor are we addicted to our phones. As such the first advice is the most suitable of them all because it applies to all ages and parts of life.
  \item An advice most important for students and people learning or experiencing new things is to always ask questions if something is not clear to you or bring up concerns about situations or behaviours. We think not asking about issues is the main hurdle for most students on their way to adulthood and is a lesson often overlooked.
\end{enumerate}

\subsection{Excercise 5: Life's Common Core in 20 years}
Explanation: Brainstorm on what Life's Common Core requirements parents or teachers might come up with in twenty years' time. Explain what might have changed and what might have become necessary by then.
Requirements: written, 300 words minimum, in-class

Due to technological and mechanical advancements in science and consumer               technolgies the common life advice will change dramatically in the next two decades even more so than it already changed greatly in the last four decades due to the rise of computers. Teachers and parents will have to adapt to it and incorporate computers into lessons and the whole education system, otherwise they won't be able to keep up. Another thing to change in the next two decades is currency. I think in 15 years currency will be completely digitally managed and handled. This means parents will have to find new strategies to teach money management and frugality to their kids as it will not be the same as with physical money just because of psychological effects. Also parents should be able to code properly by then as it will be a major part of common life in 20 years in my view. Schools will have to push more into technologies, especially in the field of computer sciences and teach programming more efficiently and more contemporary. Coding or programming will be a necessary skill to have as it serves as a way to maintain freedom of speech and personal autonomy for the individual person. Both of these things will rise in significance as the outreach of technology and especially the internet in todays form grows into political and world economical areas like elections or wallstreet. Citizens of big goverments like the United States of America will need the skills of protecting themselves from cyber attacks from within and outside of their own government and in two decades this can only be achieved by knowing how to code. Parents as well as teachers will stick to the same main principles as parents and teachers today or 50 years ago but they will need to incorporate technologies more or they \textit{will} fall behind the times.

\subsection{Summarization}
A british person has their 16th birthday and sits with their mom for breakfast. They then receive a text with cypher messages from a friend telling them to meet up at a Starbucks at 11am. The person deosn't want to go but feels the need to anyway. They want to be three or four years old, not having any responsibility again. The person and their friend meet up and want to buy a pregnancy test because the girl thinks she's pregrant, they don't have enough money for it though. As the friend goes home to get more money the first person spends their money at the Starbucks and goes home with their phone turned off.

\subsection{Narrator's thoughts/feelings}
The Narrator feels overwhelmed and doesn't feel like he can handle the situation.
On line he says "[...]I didn't want to be sixteen[...]. I wanted to be three or four[...].". He thinks he can't cope with these things and doesn't want to take responsibility over it.



\end{document}
