%------------------------------------------------------------------------------
\documentclass[11pt]{article}

\usepackage[utf8]{inputenc}
\usepackage[hyphens]{url}
\usepackage{hyperref}
\usepackage{xcolor}
\usepackage{textcomp}
\usepackage[T1]{fontenc}
\usepackage[a4paper,width=150mm,top=25mm,bottom=25mm,bindingoffset=6mm]{geometry}

\title{\textbf{Einreichung des Themas zur Themenpräsentation im Seminarfach der Klassenstufe 11}}

\definecolor{myGreen}{RGB}{97, 154, 76}
\definecolor{myRed}{RGB}{176, 72, 72}
\definecolor{myBlue}{RGB}{77, 130, 185}
\definecolor{myOrange}{RGB}{221, 137, 51}

\setlength{\parindent}{0em}
\setlength{\parskip}{0em}

%------------------------------------------------------------------------------

\begin{document}

\maketitle

\medskip

\textbf{Namen:} \textit{Janice Greven, Luise Reuchsel und Marius Cramer}

\bigskip

\textbf{Thema:} \textit{Entwicklung eines Habitatmodells für den Mars nach realistischen\newline Bedingungen und Vorraussetzungen}

\bigskip

\textbf{Fachbetreuer:} \textit{Univ.-Prof. Dr. rer. nat. habil.
Siegfried Stapf}

\bigskip

\textbf{Kurzexposé:} \textit{Wir möchten eine Seminarfacharbeit rund um das Thema Habitat auf dem Mars schreiben. Sie wird sich mit den vorzufindenden Bedingungen auf dem Mars bezüglich Atmosphäre, Boden und Planet sowie den lebenswichtigen Technologien und am Ende eine eigene Entwicklung eines Habitatmodells befassen. Lebenswichtige Technologien sind zum Beispiel Energie-, Wasser-, Nahrungs- und Sauerstoffgewinnung.} \par
\textit{Unser Ziel ist es ein Modell zu entwickeln, welches die Bedingungen auf dem Mars mit den notwendigen Mitteln zum Überleben von Menschen auf diesem fremden Planet vereint. Durch die vielen Sonden und Rover welche dem Mars untersuchen gibt es schon einige, wenige Studien zu dem Thema, allerdings ist es noch lange nicht in der Populärwissenschaft angekommen und dazu möchten wir mit unserer Arbeit beitragen. Trotz der Komplexität der Thematik möchten wir hiermit dieses interessante Gebiet der Wissenschaft an unsere Schule bringen. Unsere Hauptquellen werden Studien und Forschungen in diesem Bereich sein, vorallem von der NASA, ESA und eventuell ROSCOSMOS. Außerdem können uns Bücher und populär-wissenschaftliche Artikel dabei helfen wichtige Informationen zu erarbeiten. Prof. Stapf hat zugestimmt uns als Außenbetreuer zu unterstützen. Er ist der leitende Professor der Abteilung 'Technische Physik II' der TU Ilmenau, er leitet auch ein Projekt, welches sich mit dem Mars und den physischen Eigenschaften beschäftigt. Weiterhin haben wir Kontakt zu Dr. Christiane Heinicke aufgenommen, sie hat an dem Projekt HI-SEAS von der NASA teilgenommen, welches 8 Monate lang ein Habitat auf dem Mars simuliert hat. Sie werden uns mit ihrer Erfahrung in genau dem Gebiet, das wir untersuchen möchten weiterhelfen und unterstützen können. Wir möchten die Ausarbeitungsphase möglichst früh anfangen um die Arbeit gut verteilen zu können und somit am Ende noch genug Zeit für die Entwicklung des Habitatmodells und die Planung des Kolloqiuums zu haben.} \par
\textit{Unser Endziel ist es, ein ideales Habitat als Modell zu entwickeln und zu bauen. Dies wird sich nach den ausgearbeiteten Informationen, Bedingungen und Nöten der zukünftigen Weltraumpioniere richten und all diese in Betracht ziehen. Da wir alle drei sehr interessiert an dem Thema sind, denken wir, dass wir durchaus in der Lage sind eine wissenschaftliche Arbeit in diesem Ausmaß zu schreiben, welche uns selber zufrieden stellen wird.}
\end{document}
