%-----------------------------------------------------------------------
\documentclass[12pt]{article}

\usepackage[utf8]{inputenc}
\usepackage[hyphens]{url}
\usepackage{hyperref}
\usepackage[backend=biber]{biblatex}
\usepackage{xcolor}
\usepackage{textcomp}
\usepackage[T1]{fontenc}
\usepackage[a4paper,width=150mm,top=25mm,bottom=25mm,bindingoffset=6mm]{geometry}
\addbibresource{sources.bib}

\title{Entwicklung eines Habitat  modells für Mars nach realistischen Bedingungen und Vorraussetzungen}

\definecolor{myGreen}{RGB}{97, 154, 76}
\definecolor{myRed}{RGB}{176, 72, 72}
\definecolor{myBlue}{RGB}{77, 130, 185}
\definecolor{myOrange}{RGB}{221, 137, 51}

\setlength{\parindent}{1em}
\setlength{\parskip}{1em}

\renewcommand*\contentsname{Inhaltsangabe}

%-----------------------------------------------------------------------
\begin{document}

\begin{titlepage}
  \begin{center}
      \vspace*{1cm}

      \huge
      \textbf{Entwicklung eines Habitatmodells für Mars nach realistischen Bedingungen und Vorraussetzungen}

      \vspace{1.5cm}

      \large
      Janice Greven, Luise Reuchsel und Marius Cramer

      \vfill

      \large
      \textbf{Interner Workplan}

      \vspace{0.8cm}


      \normalsize
      KGS Erfurt\\
      Deutschland\\
      16. August, 2018

  \end{center}
\end{titlepage}


\newpage

\maketitle

\medskip

\tableofcontents

\newpage

\section*{\textbf{Vorpräsentation}}
\begin{itemize}
  \item Thema
  \item kurzer inhaltlicher Abriss
  \item Wie sind wir zum Thema gekommen?
  \item Methoden
  \item Mögliche Ansprechpartner
  \item alle Schüler präsentieren
  \item nur Overhead/Folien und Plakate/Bilder
\end{itemize}

\section{\textbf{Vorpräsentation}}
\begin{enumerate}
  \item Wie sind wir zum Thema gekommen?
  \begin{itemize}
    \item Ich habe das Thema vorgeschlagen, da ich mich schon seit einiger Zeit mit Raumfahrt, Astrophysik und Kosmologie beschäftige. Unser eigentliches Thema war ein Vergleich der Habitatmöglichkeiten auf Mars, allerdings waren Luise und Janice mehr in Psychologie bzw. Biochemie interessiert, weswegen wir das mit in unsere Arbeit einbauen werden.
  \end{itemize}
  \item Bedeutung des Themas?
  \begin{itemize}
    \item Da Mars in wenigen Jahren unsere Zukunft sein wird, müssen wir uns mehr mit ihm auseinandersetzen und auch als Jugend offen für neue Visionen sein. Aus diesem Grund haben wir uns dazu entschieden, eine Arbeit, welche im Pionierwesen unserer Zukunft forscht, zu schreiben.
  \end{itemize}
  \item Weshalb lohnt sich eine Auseinandersetzung?
  \begin{itemize}
    \item Eine Seminarfacharbeit in diesem Themenbereich lohnt sich sehr, da 1. nach unserem Wissen noch keine Arbeit zu diesem Thema geschrieben wurde an unserer Schule und möglicherweise in ganz Erfurt. Außerdem ist es ein sehr interessantes Thema, welches mit jedem Jahr immer wichtiger werden wird, für die ganze Menschheit.
  \end{itemize}
  \item Ziel?
  \begin{itemize}
    \item Unser Ziel ist, eine Arbeit zu schreiben, welche forscht, aber auch schon geschriebene wissenschaftliche Aufsätze auswertet und vergleicht sowie eigene Modelle und Ideen einfügt und entwickelt.
  \end{itemize}
\end{enumerate}

\medskip

\section{\textbf{Thesenpapier}}
Äußere Struktur des Thesenpapiers:
\begin{itemize}
  \item \textbf{DIN-A4-Format}, nicht mehr als \textbf{eine Seite}
  \item maschinengeschrieben bzw. \textbf{Computerausdruck}
  \item Entsprechend seiner Funktion eine Stellungnahme zu sein, benötigt ein Thesenpapier Angaben wie:
  \begin{itemize}
    \item \textbf{Thesen zu...}
    \item \textbf{Vorgelegt von...}
    \item \textbf{Ort, Datum, ggf. Veranstaltung}
  \end{itemize}
\end{itemize}

\section{\textbf{Ausarbeitung}}

\medskip

\section{\textbf{Themenbegrenzung}} % TODO: Themen aufteilen!
\begin{enumerate}
  \item Janice Greven
  \begin{itemize}
    \item %%
  \end{itemize}
  \item Luise Reuchsel
  \begin{itemize}
    \item %%
  \end{itemize}
  \item Marius Cramer
  \begin{itemize}
    \item %%
  \end{itemize}
\end{enumerate}

\medskip

\section{\textbf{Aufbau}}
\begin{enumerate}
  \item \textbf{Teil 1: natürliche Bedingen auf Mars} (atmosphärische Bedingen, Bodenbedingen, Planetare Bedingen)
  \item \textbf{Teil 2: lebenswichtige Technologien} (Sauerstoff-, Nahrungs- und Wassergewinnung, Müllentsorgung, Energiegewinnung)
  \item Vergleich von Habitaten unter Ingenieurwesen
  \item Vergleich von Habitaten unter Bezug auf den Menschen
  \item \textbf{Teil 3: Entwicklung eines idealen Habitatmodells}
\end{enumerate}

\medskip

\section{\textbf{Baumaterialien}(auf Mars)} % TODO: Materialien ausarbeiten

\medskip

\section{\textbf{Baumaterialien}(Modell)} % TODO: Materialien ausarbeiten

\newpage

\section{\textbf{Vergleichsaspekte}}
\begin{enumerate}
  \item Kosten
  \item Lebensraumgröße
  \item physikalische Möglichkeit der Bauform
  \item Transport
  \item Baumaterialien
\end{enumerate}

\medskip

\section{\textbf{Thesen}} % TODO: Thesen aufstellen
\begin{enumerate}
  \item Atomkraft ist die beste Energiequelle auf Mars.
  \item Ein Habitat sollte aus auf Mars vorkommenden Materialien gebaut werden um Transportkosten zu sparen. \cite{rabbithole2010a}
  \item %%
  \item %%
  \item %%
  \item %%
\end{enumerate}

\medskip

\section{\textbf{Besprechungspunkte}}
\begin{itemize}
  \item Aufbau der Arbeit
  \item Energiegewinnung
  \item Quellenarbeit
  \item
  \item
\end{itemize}

\medskip

\printbibliography[title=Referenzen]

\end{document}
